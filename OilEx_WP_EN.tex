%%%%%%%%%%%%%%%%%%%%%%%%%%%%%%%%%%%%%%%%%
% The Legrand Orange Book
% LaTeX Template
% Version 2.3 (8/8/17)
%
% This template has been downloaded from:
% http://www.LaTeXTemplates.com
%
% Original author:
% Mathias Legrand (legrand.mathias@gmail.com) with modifications by:
% Vel (vel@latextemplates.com)
%
% License:
% CC BY-NC-SA 3.0 (http://creativecommons.org/licenses/by-nc-sa/3.0/)
%
% Compiling this template:
% This template uses biber for its bibliography and makeindex for its index.
% When you first open the template, compile it from the command line with the 
% commands below to make sure your LaTeX distribution is configured correctly:
%
% 1) pdflatex main
% 2) makeindex main.idx -s StyleInd.ist
% 3) biber main
% 4) pdflatex main x 2
%
% After this, when you wish to update the bibliography/index use the appropriate
% command above and make sure to compile with pdflatex several times 
% afterwards to propagate your changes to the document.
%
% This template also uses a number of packages which may need to be
% updated to the newest versions for the template to compile. It is strongly
% recommended you update your LaTeX distribution if you have any
% compilation errors.
%
% Important note:
% Chapter heading images should have a 2:1 width:height ratio,
% e.g. 920px width and 460px height.
%
%%%%%%%%%%%%%%%%%%%%%%%%%%%%%%%%%%%%%%%%%

%----------------------------------------------------------------------------------------
%	PACKAGES AND OTHER DOCUMENT CONFIGURATIONS
%----------------------------------------------------------------------------------------

\documentclass[11pt,fleqn]{book} % Default font size and left-justified equations

%----------------------------------------------------------------------------------------

\input{structure} % Insert the commands.tex file which contains the majority of the structure behind the template

\begin{document}

%----------------------------------------------------------------------------------------
%	TITLE PAGE
%----------------------------------------------------------------------------------------

\begingroup
\thispagestyle{empty}
%\vfill
%\begin{center}
%    \includegraphics[scale=0.7]{logo}
%\end{center}
\begin{tikzpicture}[remember picture,overlay]
    \node(background) at (current page.center) {\includegraphics[width=\paperwidth]{background}};
	\draw (current page.center) node (title)[fill=green(ryb),fill opacity=0.6,text opacity=1,inner sep=1cm]{\Huge\centering\bfseries\sffamily\parbox[c][][t]{\paperwidth}{\color{darkgoldenrod}\centering OILEX\\[15pt]{\Large UCO Exchange Marketplace}\\[20pt]}};
\end{tikzpicture}
\vfill
\vfill
\begin{center}
	\color{darkgoldenrod} \LARGE WHITEPAPER \\
	\color{darkgoldenrod} \large www.oilex.io
\end{center}
\endgroup

%----------------------------------------------------------------------------------------
%	COPYRIGHT PAGE
%----------------------------------------------------------------------------------------

\newpage
~\vfill
\thispagestyle{empty}

\noindent Copyright \copyright\ 2013 John Smith\\ % Copyright notice

\noindent \textsc{Published by Publisher}\\ % Publisher

\noindent \textsc{book-website.com}\\ % URL

\noindent Licensed under the Creative Commons Attribution-NonCommercial 3.0 Unported License (the ``License''). You may not use this file except in compliance with the License. You may obtain a copy of the License at \url{http://creativecommons.org/licenses/by-nc/3.0}. Unless required by applicable law or agreed to in writing, software distributed under the License is distributed on an \textsc{``as is'' basis, without warranties or conditions of any kind}, either express or implied. See the License for the specific language governing permissions and limitations under the License.\\ % License information

\noindent \textit{First printing, March 2013} % Printing/edition date

%----------------------------------------------------------------------------------------
%	TABLE OF CONTENTS
%----------------------------------------------------------------------------------------

%\usechapterimagefalse % If you don't want to include a chapter image, use this to toggle images off - it can be enabled later with \usechapterimagetrue

\chapterimage{chapter_head_1.pdf} % Table of contents heading image

\pagestyle{empty} % No headers

\tableofcontents % Print the table of contents itself

\cleardoublepage % Forces the first chapter to start on an odd page so it's on the right

\pagestyle{fancy} % Print headers again

%----------------------------------------------------------------------------------------
%	PART
%----------------------------------------------------------------------------------------

\part{Part One}

%----------------------------------------------------------------------------------------
%	CHAPTER 1
%----------------------------------------------------------------------------------------

\chapterimage{chapter_head_2.pdf} % Chapter heading image

\chapter{Text Chapter}

\section{Abstract}\index{Abstract}
The World is faced with a formidable challenge. The hope of the Kyoto Protocol was that
industrialised nations would cut their collective emissions of greenhouse gases by 5.2\% in
2012 compared to the year 1990. However, that World governments was
not capable of achieving even this token first-step. And this at a time when all manner of
climate-induced disasters are gathering force, and polar ice caps melting at a rate that even
the worst-case scenario computer models did not anticipate.
Action is needed and action is needed now. Road transport contributes about 20\% of the
EU27 carbon dioxide emissions, and must play a fundamental role in greenhouse gas
emissions reductions if we are to stave off catastrophic impacts on our living planet. Despite
the desperate urgency of the situation, alternative fuel producers and environmental
technologies are constantly facing resistance, criticism and hostility, which, if the same strict
criteria were applied to the present generation of energy technologies, it would make a
mockery of them. Biodiesel has not escaped such harsh and ill-informed attacks, particularly
in the last year.
It is the aim of this project to give in-depth information to prospective biodiesel producers,
or project managers, to enable correct decision-making and to ensure success for their
proposed projects. It seeks to analyse the real potential in the EU27 for biodiesel production
from Used Cooking Oil (UCO), and its place in the market.

%\lipsum[1-7] % Dummy text

%------------------------------------------------

\section{Introduction}\index{Introduction}
Biodiesel is an alternative road fuel made from transesterfied fatty acids. The most common
form is made from straight vegetable oil, whether that be rapeseed oil, soya oil or others. It
can however, also be produced from used cooking oil (UCO) or animal fat such as tallow,
and if processed correctly will produce high quality fuel. Although certain diesel engines can
run on straight vegetable oil, if transformed into biodiesel it can be used in almost all diesel
engines, most importantly in modern high-performance direct injection engines.
Despite the knowledge of the possibilities for running diesel engines on vegetable oil having
been overlooked for a large part of the last century, Dr Rudolf Diesel first developed the
Diesel engine in 1895 with the full intention of running it on a variety of fuels, including
vegetable oil. The concept is neither revolutionary nor fanciful.
Recent developments at European Union level are transforming both the disposal method of
Used Cooking Oil (UCO) and the way in which the EU fuels its road transport vehicles.
These combined developments have made the use and production of biodiesel from UCO an
increasingly favourable prospect.
In May 2003, the European Parliament and the Council adopted the 'Directive on the
promotion of the use of biofuels or other renewable fuels for transport'. This Directive
requires that Member States in 2005 to replace 2\% of their diesel and petrol with biofuels,
and replacing 5,75 \% by 2010.
The EU Animal by-product Regulation 1774/2002 sets restrictions on the use of Used
Cooking Oil originating in restaurants, catering facilities and kitchens. The effect is that,
except for in special cases, UCO from catering premises can no longer be used as an
ingredient in animal feed, which historically was its main disposal route. In parallel, the
Landfill Directive 99/31EC requires each Member State to set out a pollution control regime
which prohibits the acceptance of certain types of wastes at landfills, including liquid wastes
4
such as UCO. Furthermore, recent statistics show a huge increase in volume of production of
UCO in the last few decades and the number of catering establishments in European
countries is on the increase.
With a growing amount of UCO in the EU, the disposal problems UCO generators now face,
and the concern to remove the UCO from the food chain, the production of biodiesel offers
an ideal solution. The waste management exigency and sustainable transport strategy can
both be addressed by the production of biodiesel. This is a real opportunity.

%This statement requires citation \cite{article_key}; this one is more specific \cite[162]{book_key}.

%------------------------------------------------

\section{Key Terminology}\index{Key Terminology}
\begin{description}
	\item[UCO]: Used Cooking Oil. Also referred to throughout the literature as WVO (waste vegetable
oil) and UVO (used vegetable oil), RVO (recycled vegetable oil) and RCO (Recycled Cooking
Oil). UCO has been chosen as the standard for this document.
    \item[FAME]: Fatty Acid Methyl Ester. The technical acronym for biodiesel.
    \item[UCOME]: Used Cooking Oil Methyl Ester. FAME coming from UCO.
	\item[B30]: The use of B followed by a number such as B30 and B100 is used in this document
where appropriate, to identify the various percentages of biodiesel blends. Although
historically mainly used in America, the terminology is well understood and it is starting to be
used across Europe now too, hence was deemed appropriate.
    \item[Billion]: The Anglo-Saxon use of billion has been chosen. That is to say for the purposes of
this document, it represents one thousand million, and not a million million, which is referred
to as a trillion.
    \item[“,” or “.”]: The Anglo-Saxon use of decimal points and commas to represent numbers has
been chosen for the purposes of this document (except in the accompanying excel
spreadsheets, and the occasional figure or table derived from excel, where continental
numeration has been used). That is to say that a “.” represents a decimal point, and a “,”
distinguishes between multiples of thousands, millions and billions.
[References]: Where sources have not been given, the information has come directly from
OilEx consortium partners.
    \end{description}

%Lists are useful to present information in a concise and/or ordered way\footnote{Footnote example...}.

\subsection{Background}\index{Background}
OilEx project is a project to facilitate the uptake of used cooking oil to produce biodiesel. The objective of OilEx is the promotion of localised biodiesel production for transportation purposes, by means of the active involvement of key local actors the European and Extra-European countries that want to participate.

The promotion will be enforced by the creation of a UCO exchange marketplace powered by Blockchain Technology and by the creation of a OCF (OilEx Community Fund) to promote such initiatives.

There are essentially four phases to the OilEx project:
\begin{description}
	\item[Information gathering and synthesis (WP2)] This phase recognises the existence of multiple sources of information and experience from
    across Europe concerning the supply chain of biodiesel UCOME. It aims to make the first
    comprehensive analysis of this state of the art to form the basis of the later project phases.    
	\item[The development of tools and resources (WP3)] The second phase takes the results of the first and develops from them a set of tools and
    resources which provide concise and comprehensible guidance to market actors in any
    Member State. With this guidance new biodiesel production facilities can be initiated and
    vehicle fleets converted to biodiesel.    
    \item[The set up of demonstration activities (WP 4-6)] Using the tools and resources developed in WP3, Work packages 4-6 focus on bringing
    collected knowledge and tools into practice. The three work packages reflect three major
    focal points (and target groups) within the supply chain for establishing successful biodiesel
    demonstrations on local scale: production of local biodiesel plants (WP4), distribution
    facilities for biodiesel (WP5), and demand development for fleets (WP6). The demonstration
    phase forms the heart of the OilEx action; WP 2 and 3 are focused on providing
    deliverables (e.g. tools) that enable successful and efficient demonstration activities.    
    \item[Dissemination (WP 7/8) and Project Coordination (WP1)] During the full duration of the project, dissemination activities (WP 7/8) are carried out in
    which results from the individual work packages are disseminated to relevant target groups
    including project partners, OilEx supporters, EC delegates as well as relevant target
    groups. This phase covers a wide range of dissemination techniques, from printed and
    electronic handbooks to workshops and training sessions, ongoing networks, all having the
    ultimate goal of increasing the uptake of biodiesel among public and private transport fleets
    across the EU. An overarching work package is concerned with the management of the
    project from start to finish, ensuring proper coordination, quality assurance and budgetary
    control (WP1).
\end{description}

\subsection{Bullet Points}\index{Lists!Bullet Points}

\begin{itemize}
	\item The first item
	\item The second item
	\item The third item
\end{itemize}

\subsection{Descriptions and Definitions}\index{Lists!Descriptions and Definitions}

\begin{description}
	\item[Name] Description
	\item[Word] Definition
	\item[Comment] Elaboration
\end{description}

%----------------------------------------------------------------------------------------
%	CHAPTER 2
%----------------------------------------------------------------------------------------

\chapter{In-text Elements}

\section{Theorems}\index{Theorems}

This is an example of theorems.

\subsection{Several equations}\index{Theorems!Several Equations}
This is a theorem consisting of several equations.

\begin{theorem}[Name of the theorem]
	In $E=\mathbb{R}^n$ all norms are equivalent. It has the properties:
	\begin{align}
		 & \big| ||\mathbf{x}|| - ||\mathbf{y}|| \big|\leq || \mathbf{x}- \mathbf{y}||                            \\
		 & ||\sum_{i=1}^n\mathbf{x}_i||\leq \sum_{i=1}^n||\mathbf{x}_i||\quad\text{where $n$ is a finite integer}
	\end{align}
\end{theorem}

\subsection{Single Line}\index{Theorems!Single Line}
This is a theorem consisting of just one line.

\begin{theorem}
	A set $\mathcal{D}(G)$ in dense in $L^2(G)$, $|\cdot|_0$.
\end{theorem}

%------------------------------------------------

\section{Definitions}\index{Definitions}

This is an example of a definition. A definition could be mathematical or it could define a concept.

\begin{definition}[Definition name]
	Given a vector space $E$, a norm on $E$ is an application, denoted $||\cdot||$, $E$ in $\mathbb{R}^+=[0,+\infty[$ such that:
	\begin{align}
		 & ||\mathbf{x}||=0\ \Rightarrow\ \mathbf{x}=\mathbf{0}        \\
		 & ||\lambda \mathbf{x}||=|\lambda|\cdot ||\mathbf{x}||        \\
		 & ||\mathbf{x}+\mathbf{y}||\leq ||\mathbf{x}||+||\mathbf{y}||
	\end{align}
\end{definition}

%------------------------------------------------

\section{Notations}\index{Notations}

\begin{notation}
	Given an open subset $G$ of $\mathbb{R}^n$, the set of functions $\varphi$ are:
	\begin{enumerate}
		\item Bounded support $G$;
		\item Infinitely differentiable;
	\end{enumerate}
	a vector space is denoted by $\mathcal{D}(G)$.
\end{notation}

%------------------------------------------------

\section{Remarks}\index{Remarks}

This is an example of a remark.

\begin{remark}
	The concepts presented here are now in conventional employment in mathematics. Vector spaces are taken over the field $\mathbb{K}=\mathbb{R}$, however, established properties are easily extended to $\mathbb{K}=\mathbb{C}$.
\end{remark}

%------------------------------------------------

\section{Corollaries}\index{Corollaries}

This is an example of a corollary.

\begin{corollary}[Corollary name]
	The concepts presented here are now in conventional employment in mathematics. Vector spaces are taken over the field $\mathbb{K}=\mathbb{R}$, however, established properties are easily extended to $\mathbb{K}=\mathbb{C}$.
\end{corollary}

%------------------------------------------------

\section{Propositions}\index{Propositions}

This is an example of propositions.

\subsection{Several equations}\index{Propositions!Several Equations}

\begin{proposition}[Proposition name]
	It has the properties:
	\begin{align}
		 & \big| ||\mathbf{x}|| - ||\mathbf{y}|| \big|\leq || \mathbf{x}- \mathbf{y}||                            \\
		 & ||\sum_{i=1}^n\mathbf{x}_i||\leq \sum_{i=1}^n||\mathbf{x}_i||\quad\text{where $n$ is a finite integer}
	\end{align}
\end{proposition}

\subsection{Single Line}\index{Propositions!Single Line}

\begin{proposition}
	Let $f,g\in L^2(G)$; if $\forall \varphi\in\mathcal{D}(G)$, $(f,\varphi)_0=(g,\varphi)_0$ then $f = g$.
\end{proposition}

%------------------------------------------------

\section{Examples}\index{Examples}

This is an example of examples.

\subsection{Equation and Text}\index{Examples!Equation and Text}

\begin{example}
	Let $G=\{x\in\mathbb{R}^2:|x|<3\}$ and denoted by: $x^0=(1,1)$; consider the function:
	\begin{equation}
		f(x)=\left\{\begin{aligned}    & \mathrm{e}^{|x|} &  & \text{si $|x-x^0|\leq 1/2$} \\
			 & 0                &  & \text{si $|x-x^0|> 1/2$}\end{aligned}\right.
	\end{equation}
	The function $f$ has bounded support, we can take $A=\{x\in\mathbb{R}^2:|x-x^0|\leq 1/2+\epsilon\}$ for all $\epsilon\in\intoo{0}{5/2-\sqrt{2}}$.
\end{example}

\subsection{Paragraph of Text}\index{Examples!Paragraph of Text}

\begin{example}[Example name]
	\lipsum[2]
\end{example}

%------------------------------------------------

\section{Exercises}\index{Exercises}

This is an example of an exercise.

\begin{exercise}
	This is a good place to ask a question to test learning progress or further cement ideas into students' minds.
\end{exercise}

%------------------------------------------------

\section{Problems}\index{Problems}

\begin{problem}
What is the average airspeed velocity of an unladen swallow?
\end{problem}

%------------------------------------------------

\section{Vocabulary}\index{Vocabulary}

Define a word to improve a students' vocabulary.

\begin{vocabulary}[Word]
	Definition of word.
\end{vocabulary}

%----------------------------------------------------------------------------------------
%	PART
%----------------------------------------------------------------------------------------

\part{Part Two}

%----------------------------------------------------------------------------------------
%	CHAPTER 3
%----------------------------------------------------------------------------------------

\chapterimage{chapter_head_1.pdf} % Chapter heading image

\chapter{Presenting Information}

\section{Table}\index{Table}

\begin{table}[h]
	\centering
	\begin{tabular}{l l l}
		\toprule
		\textbf{Treatments} & \textbf{Response 1} & \textbf{Response 2} \\
		\midrule
		Treatment 1         & 0.0003262           & 0.562               \\
		Treatment 2         & 0.0015681           & 0.910               \\
		Treatment 3         & 0.0009271           & 0.296               \\
		\bottomrule
	\end{tabular}
	\caption{Table caption}
\end{table}

%------------------------------------------------

\section{Figure}\index{Figure}

\begin{figure}[h]
	\centering\includegraphics[scale=0.5]{placeholder}
	\caption{Figure caption}
\end{figure}

%----------------------------------------------------------------------------------------
%	BIBLIOGRAPHY
%----------------------------------------------------------------------------------------

\chapter*{Bibliography}
\addcontentsline{toc}{chapter}{\textcolor{ocre}{Bibliography}}

%------------------------------------------------

\section*{Articles}
\addcontentsline{toc}{section}{Articles}
\printbibliography[heading=bibempty,type=article]

%------------------------------------------------

\section*{Books}
\addcontentsline{toc}{section}{Books}
\printbibliography[heading=bibempty,type=book]

%----------------------------------------------------------------------------------------
%	INDEX
%----------------------------------------------------------------------------------------

\cleardoublepage
\phantomsection
\setlength{\columnsep}{0.75cm}
\addcontentsline{toc}{chapter}{\textcolor{ocre}{Index}}
\printindex

%----------------------------------------------------------------------------------------

\end{document}
